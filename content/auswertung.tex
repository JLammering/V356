\section{Auswertung}
\label{sec:Auswertung}

\subsection{Durchlasskurve}

\subsubsection{Berechnung}

\begin{table}
  \centering
  \caption{Tabelle mit den gegebenen Werten.}
  \label{tab:wertedurch}
  \begin{tabular}{c c c}
    \toprule
    $L \ /\ \si{\milli\henry}$ & $C_1 \ /\ \si{\nano\farad}$ & $C_2 \ /\ \si{\nano\farad}$ \\
    \midrule
    1,217 & 20 & 10\\
    \bottomrule
  \end{tabular}
\end{table}

Es existieren drei Grenzfrequenzen. Da es ein Tiefpass ist, werden alle Spannungen
dessen Frequenz unterhalb eines bestimmten Wertes liegen durchgelassen.
In diesem Bereich fließt der Strom über die Spulen und kann so den Stromkreis
passieren. Dieser Wert
wird folgend $\nu_1$ genannt und ist die erste Grenzfrequenz.
Zur Berechnung wird die Formel \eqref{nu1} genutzt.

\begin{equation}
  \nu_1 = \frac{1}{2\cdot\pi}\sqrt{\frac{2}{L*C_1}}
\end{equation}
In diesem Fall ergibt dies $\nu_1 = \SI{45622}{\hertz}$.

Ab $\nu_1$ existiert ein Frequenzbereich in dem kaum Spannung durchgelassen wird.
Der geht bis $\nu_2$ der zweiten Grenzfrequenz. Die Formel lautet :

\begin{equation}
  \nu_2 = \frac{1}{2\cdot\pi}\sqrt{\frac{2}{L*C_2}}.
\end{equation}
Das Ergebnis ist $\nu_2 = \SI{64519}{\hertz}$.

Bis $\nu_3$, der dritten Grenzfrequenz, ist ein Bereich, in dem wieder Spannung
durchgelassen wird. Und ab da wird endgültig keine Spannung durchgelassen. Die
benutzte Formel lautet:

\begin{equation}
  \nu_1 = \frac{1}{2\cdot\pi}\sqrt{\frac{2*(C_1+C_2)}{L*C_1*C_2}}.
\end{equation}
Das Ergebnis ist $\nu_3 = \SI{79020}{\hertz}$.

\subsubsection{Aufgenommene Messkurve}


\subsection{Dispersionskurve}

\subsubsection{Messwerte}

\begin{table}
  \centering
  \caption{Tabelle mit den Messwerten.}
  \label{tab:wertedis}
  \begin{tabular}{c c}
    \toprule
     $\nu \ /\ \si{\hertz}}$ & Phasenverschiebung\\
    \midrule
    4 & 0\\
    4360 & 1\\
    7767 & 2\\
    12102 & 3\\
    15225 & 4\\
    19075 & 5\\
    23271 & 6\\
    26975 & 7\\
    30055 & 8\\
    33298 & 9\\
    35276 & 10\\
    38600 & 11\\
    44560 & 13\\
    52613 & 16\\
    \bottomrule
  \end{tabular}
\end{table}

In Tabelle \ref{tab:wertedis} sind die gemessenen Werte zur Bestimmung der
Dispersionskurven abgebildet. Hier wurde anders als bei der Kopie der
Originaldaten angenommen, dass zwischen dem elften und zwölften Messwert,
sowie dem zwölften und dreizehnten Wert ein Wert fehlt.

\subsubsection{Grafiken}

\begin{figure}
  \centering
  \includegraphics[width = \textwidth]{build/plotw1.pdf}
  \caption{Der untere Ast der Dispersionskurve.}
  \label{fig:astw1}
\end{figure}

In Abbildung \ref{fig:astw1} sind die aufgenommenen Messwerte, sowie zwei
Theoriekurven eingezeichnet.

Die Kurven $\omega_1$ und $\omega_1(\frac{\pi}{2}$ sind nach Gleichung
\eqref{eqn:Omega} mit dem negativen
Vorzeichen mit $matplotlib$ \cite{matplotlib} erstellt worden.
$\omega_1(\frac{\pi}{2}$ ist die Kreisfrequenz die durch Division durch
$2\cdot\pi$ die Grenzfrequenz $\nu_1$ ergibt.

\begin{figure}
  \centering
  \includegraphics[width = \textwidth]{build/plotw2.pdf}
  \caption{Der obere Ast der Dispersionskurve.}
  \label{fig:astw2}
\end{figure}

In Abbildung \ref{fig:astw2} sind drei Theoriekurven eingezeichnet.
