\section{Diskussion}
\label{sec:Diskussion}

\subsection{Durchlasskurve}

Die aus dem Graphen bestimmten Grenzfrequenzen stimmen in guter Näherung mit
den aus der Theorie berechneten überein. $\nu_1$ weicht \SI{5.4}{\percent} vom
Theoriewert ab. $\nu_2$ weicht \SI{0.21}{\percent} ab und $\nu_3$ weicht \SI{0.1}{\percent}
ab. Die großen Schwankungen besonders im kleineren Frequenzbereich sind darauf
zurückzuführen, dass wir den Wellenwiderstand $Z$ als konstant angenommen haben und
wir ihn bei $\omega = \SI{0}{\hertz}$ berechnet haben. In der Nähe der Grenzfrequenz
steigt er aber stark an, ist also frequenzabhängig.

\subsection{Dispersionskurven}

Die Messwerte liegen nahe der Kurve $\omega_1$. Erst nahe der Grenzfrequenz
$\omega_1(\frac{\pi}{2})$ weichen sie stärker ab. Auch das kann mit dem
frequenzabhängigen Wellenwiderstand erklärt werden.

\subsection{Eigenfrequenzen}

Die Messwerte stimmen erfreulich gut mit der Theoriekurve überein. Also konnte die
Theorie bestätigt werden.

\subsection{Stehende Wellen}

In den Abbildungen \ref{fig:erstesmax} und \ref{fig:zweitesmax}
kann man gut stehende Wellen erkennen. In Abbildung \ref{fig:absw=wellw} ist die
stehende Welle nicht so gut erkennbar wie in den beiden anderen. Dem liegt zu Grunde,
dass die Amplitude nicht sehr stark variiert und demzufolge kleine Messunsicherheiten
sich stärker auswirken. Offenbar verhindert die Schaltung aus Abbildung \ref{fig:Abschluss}
weitestgehend eine Reflexion.
