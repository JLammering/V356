\section{Theorie}
\label{sec:Theorie}

\subsection{Kettenschaltungen mit LC-Gliedern}

Kettenschaltungen aus Kondensatoren und Induktivitäten können verschiedene
Frequenzen einer Wechselstromquelle voneinander separieren.
Der sogenannte Tiefpass, der in Abbildung x dargestellt ist, zeichnet
sich dadurch aus, dass er ab einer bestimmten Schwingungsfrequenz die Amplitude
des Wechselstroms verschwinden lässt.
Bei dem sogenannten Hochpass, der in Abbildung x abgebildet ist, verschwindet
die Amplitude, wenn die Schwingungsfrequenz gegen Null geht.
Dadurch ist es zum Beispiel möglich die unerwünschten Oberwellen eines
Wechselstromgenerators zu eliminieren, indem ein Tiefpass in die
Schaltung eingebaut wird.
Im Folgenden werden spezifische Tiefpässe mit konstanten und
alternierenden Kapazitäten analysiert.


\subsection{Dispersionsrelation einer LC-Kettenschaltung}

Über die Kirchhoffschen Gesetze kann mit Hilfe der in Abbildung x
dargestellten Ströme und Spannungen die Schwingungsgleichung einer
LC-Kettenschaltung aufgestellt werden.
Aus der Knotenregel

\begin{equation}
  \sum_{n} I_n = 0
\end{equation}

folgt

\begin{equation}
  I_n - I_{n+1} - I_{n,\text{quer}} = 0
\end{equation}

und aus der Maschenregel

\begin{equation}
  \sum_{n} U_n = 0
\end{equation}

im eingeschwungenen Zustand

\begin{equation}
  I_{n} = \frac{U_{n} - U_{n-1}}{\symup{i} \omega L}
\end{equation}

und

\begin{equation}
  I_{n,\text{quer}} = U_{n} \symup{i} \omega C.
\end{equation}

Daraus folgt durch Einsetzen und Umformen die Schwingungsgleichung

\begin{equation}
  \frac{U_{n} - U_{n-1}}{\symup{i} \omega L} - \frac{U_{n+1} - U_{n}}{\symup{i} \omega L} - \symup{i} U_{n} \omega C = 0,
\end{equation}

die mit dem Ansatz

\begin{equation}
  U_{n}(t) = U_0 \symup{e}^{\symup{i}t (\omega - n \theta)}
\end{equation}

gelöst werden kann.
Es ergibt sich die Dispersionsrelation

\begin{equation}
  \omega_k(\theta) = \sqrt{\frac{2}{LC}(1-\cos\theta)},
\end{equation}

das ist die Kreisfrequenz
$\omega_k$ in Abhängigkeit von der Phasenverschiebung $\theta$ pro Kettenglied.
Der maximale Wert für $\omega_k$ ist die Grenzfrequenz $\bar{\omega}$
mit

\begin{equation}
  \bar{\omega} = \sqrt{\frac{2}{LC}}.
\end{equation}

Hochfrequentere Schwingungen werden von dem Tiefpass herausgefiltert.


\subsection{Dispersionsrelation einer alternierenden LC-Kettenschaltung}

Mit einer alternierenden LC-Kettenschaltung ist hier ein Tiefpass gemeint,
bei dem zwei unterschiedliche Kapazitäten $C_1$ und $C_2$ alternierend
parallel in die Schaltung eingebaut wurden. Dies ist in Abbildung x
dargestellt.

\cite{sample}
